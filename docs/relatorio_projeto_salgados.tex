
\documentclass[conference]{IEEEtran}
\IEEEoverridecommandlockouts
\usepackage{cite}
\usepackage{amsmath,amssymb,amsfonts}
\usepackage{algorithmic}
\usepackage{graphicx}
\usepackage{textcomp}
\usepackage{xcolor}
\def\BibTeX{{\rm B\kern-.05em{\sc i\kern-.025em b}\kern-.08em
    T\kern-.1667em\lower.7ex\hbox{E}\kern-.125emX}}
\renewcommand\abstractname{Resumo}

\begin{document}

\title{Sistema de Gerenciamento de Vendas de Bolos e Salgados}

\author{\IEEEauthorblockN{Cauê Rodrigues Campos, Gustavo Campana Cruz do Vale, Marcus Vinícius da Cruz Maia}
\IEEEauthorblockA{\textit{Análise, Projeto e Programação Orientados a Objetos} \\
\textit{Universidade Federal de Minas Gerais (UFMG)}\\
Minas Gerais, Brasil, maio/2025}
}

\maketitle

\begin{abstract}
Este trabalho apresenta a estrutura parcial do projeto "Sistema de Gerenciamento de Vendas de Bolos e Salgados", desenvolvido no contexto da disciplina de Análise, Projeto e Programação Orientados a Objetos (APPOO). O sistema aplica os principais conceitos da Programação Orientada a Objetos com foco em modularização, herança e encapsulamento, utilizando Python, Flet e SQLite.
\end{abstract}

\begin{IEEEkeywords}
POO, SQLite, Python, Flet, UML, vendas
\end{IEEEkeywords}

\section{Introdução e Revisão}
O projeto visa implementar um sistema para gerenciamento de vendas de alimentos (bolos e salgados), com cadastro de produtos, usuários e pedidos. A motivação está na aplicabilidade real do domínio e na oportunidade de praticar os princípios fundamentais da POO. A revisão bibliográfica apoia-se nos conceitos de encapsulamento, herança, polimorfismo, modularização e estrutura MVC para construção de sistemas robustos e escaláveis.

\section{Metodologia}
A abordagem adotada organiza o sistema em camadas:
\begin{itemize}
    \item \textbf{models/} - define as entidades do sistema (Usuario, Cliente, Admin, Alimento, Bolo, Salgado, Pedido).
    \item \textbf{controllers/} - lida com lógica de negócio, como login, criação de pedidos e cadastro de alimentos.
    \item \textbf{views/} - interface gráfica construída com o framework Flet.
    \item \textbf{data/} - manipulação do banco de dados SQLite.
\end{itemize}
Além disso, utilizou-se o padrão de acesso a dados por meio da classe \texttt{BaseDados}.

\section{Resultados}
Foram desenvolvidos:
\begin{itemize}
    \item Diagrama UML com todas as classes, métodos, atributos e heranças.
    \item Interface gráfica funcional com tela de login e painel de cliente/admin.
    \item Estrutura modular de pastas e arquivos com versionamento no GitHub.
\end{itemize}

\section{Conclusão}


\section {Referências}

\end{document}
